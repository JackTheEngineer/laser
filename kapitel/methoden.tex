\section{Theorie}
\subsection{Funktionsweise eines Lasers}
Man benötigt zwei entscheidende Physikalische Aspekte, um einen Laser zu bauen.
Im folgenden wird ein Zweiniveausystem für ein Elektron mit den Zuständen
$\ket{1}$ und $\ket{2}$ genauer Untersucht. \\
Der erste Aspekt ist die induzierte Photonenemission. Wenn es ein Elektron in einem
angeregten Zustand $\ket{2}$ gibt, und es mit einem Photon interagiert, welches die
Differenzenergie der Zustände $\ket{2} - \ket{1}$ trägt, dann regt sich das
Elektron ab und emittiert dabei ein Photon welches deselbe Phase,
Polarisation und Richtung hat, wie das auf das Elektron eingefallene Photon.
Wenn es nun sehr viele Elektronen im Zustand $\ket{2}$ gibt, so sorgt der Effekt
der induzierten Emission für Photonen mit identischen Eigenschaften. Nun benötigt man
stets mehr Elektronen im Zustand $\ket{2}$ als im Zustand $\ket{1}$, damit es
zur kontinuierlichen induzierten Photoemission kommen kann.
\[
  n_{\ket{2}} > n_{\ket{1}}
\]
Diese Bedingung heißt Besetzungsinversion. Das ist der Zweite enscheidene Aspekt.
Mit einem reinen Zweiniveausystem lässt sich die Besetzungsinversion nicht
erreichen, da der Wirkungsquerschnitt zum anregen der Elektronen derselbe ist,
wie zum induzierten abregen. Man benötigt noch mindestens ein drittes Pumpniveau $\ket{3}$,
auf welches man Elektronen mit Photonen der Energie $\ket{3} - \ket{1}$ befördert, 
damit sich möglichst schnell auf das langlebigere Niveau $\ket{2}$ abregen und von
dort durch induzierte Abregung Laserlicht erzeugen.\\
Für einen voll funktionsfähigen Laser benötigt es noch zwei gekrümmte Resonatorspiegel, 
mit reflektionskoeffizienten $R_1$ und $R_2$. Im Laser
muss das Licht stets rückgekoppelt werden, sodass es eine stetige induzierte Anregung gibt.
Für die stetige Rückkopplung müssen die Radien $r_1$ und
$r_2$  die Stabilitätsbedingung erfüllen. Hierbei bezeichnet $L$ den Abstand zwischen den Spiegeln.
\[
   0 <  1 - \frac{L}{r_1} - \frac{L}{r_2} + \frac{L^2}{r_1r_2} < 1 
\]
In dem Resonanzmedium können Photonen absorbiert werden.
Die Verluste pro Strecke werden mit $\delta$ bezeichnet. Die Laseraktivität ist abhängig von den
Verlusten $\delta$ pro, davon wie groß die Bestzungsinversion $n_{\ket{2}} - n_{\ket{1}}$ ist,
und wie groß der Wirkungsquerschnitt $\sigma$ für die induzierte Emission ist.
Bei einem Durchlauf vom Licht zum ersten Spiegel, zurück zum zweiten Spiegel und wieder zum Anfangspunkt zurück
berechnet man alle Verlust- und Vervielfältigungseffekte vom Licht durch multiplikation
und erhält für die neue Lichtintensität
\[
I_1 = I_0R_1R_2\cdot \exp{2l(\sigma(n_{\ket{2}} - n_{\ket{1}}) - \delta)}
\]
Für eine Laseraktivität muss gelten $\sfrac{I_1}{I_0} > 1$, und somit
\[
   \frac{\delta}{\sigma} - \frac{ln\big(R_1R_0\big)}{2} < n_{\ket{2}} - n_{\ket{1}}
\]

\paragraph{Nd:YFL Laser}
Der im Versuch Verwendete Laser hat ein Neodym-Ion als aktives Material.
Das äußerste Elektron hat dabei vier Energieniveaus zur Verfügung stehen.
Mit Pumplicht aus einer Halbleiterdiode werden die Elektronen im Pumpmaterial
vom niedrigsten Zustand $\ket{1}$ in den Zustand $\ket{4}$ angeregt. Der Übergang des
vom Zustand $\ket{4}$ in den langlebigen Zustand $\ket{3}$ erfolgt sehr
schnell. Durch induzierte Lichtemission begeben sich die Elektronen in den kurzlebigen
Zustand $\ket{2}$, von dem sie erneut schnell in den Zustand $\ket{1}$ übergehen.
So ensteht eine grosse Besetzungsinversion, wiederum zu einer starken Lichintensität
nach einem Durchlauf führt.

\subsection{Erzeugung von Ultrakurz-Lichpulsen}
In dem vorher beschriebenen Resonator aus Spiegeln mit der Länge $L$
entstehen stehende elektromagnetische Longitudinalwellen mit den Wellenlängen
\begin{equation}
  2L = n \cdot \lambda_n
  \label{equ:resonance}
\end{equation}
mit dem Kreisfrequenzabstand $\triangle \omega = \sfrac{\pi c}{L}$. 
Im Resonator wird aber nur ein Teil aller Wellen welche die Bedingung \ref{equ:resonance}
erfüllen, verstärkt. 
Erzeugt man akustiche Schwingungen mit der Frequenz $\Omega$ in einem Quarzkristall, 
und setzt diesen Quarzkristall in die Resonatorstrecke, so werden durch periodische Transmission
der Kreisfrequenz $\omega_n \pm \Omega$. Ist dabei $\Omega = \triangle \omega$
bringt man durch Interferenzeffekte alle Longitudinalmoden in eine feste Phasenbeziehung
und erzeugt damit starke Peaks des elektromagetischen Feldes, Ultrakurzpulse.
Dabei hängt die Dauer und intensität von der Anzahl der gekoppelten Moden ab.
Je breitbandiger der Laser ist, umso mehr Moden können miteinander gekoppelt werden,
und umso kürzer ist der Peak.

\subsection{Optisch nichtlineare Materialien}
\paragraph{Lineare Optik}
In der klassischen Optik ist die Polarisation eines Dielektrikums linear proportional zur
elektrischen Feldstärke. Dabei ist $\chi_0$ die elektrische Suszeptibilität, also das Verhältnis,
wie sich die Polarisation in Abhängigkeit des elektrischen Feldes im Material ändert.
\begin{equation}
  \va{P}(\va{E}) = \epsilon_0\bigg(\sum_j \chi^{(1)}_{ij} E_j\bigg)
\end{equation}
Die partielle Differenzialgleichung in einer dimension für das elektrische Feld lautet dabei:
\begin{equation}
  \pdv[2]{E}{x} - \frac{\epsilon_0(1+\chi^{(1)})}{c}\pdv[2]{E}{t} = 0
\end{equation}
Diese Differenzialgleichung wird durch gelöst, durch ein Elektrisches Feld in Form ebener Wellen gelöst:
\begin{equation}
  E = E_0\exp{i(\omega t - k_{\omega}x)} + C
\end{equation}

\paragraph{Nichtlineare Optik}
In der nichtlinearen Optik wird die durch verschiedende Quanteneffekte beeinflusste Polarisation
eines mittels eines Taylorpolynoms approximiert.
\begin{align*}
  \va{P}(\va{E}) &= \epsilon_0\bigg(&\sum_j  \chi^{(1)}_{ij} E_j  &+  \sum_{jk} \chi_{ijk}^{(2)} E_j E_k + \sum_{jk\ell} \chi_{ijk\ell}^{(3)} E_j E_k E_\ell + \cdots \bigg)\\
  \va{P}(\va{E}) &= &\va{P_L}\ \ \ \  &+\ \ \ \ \va{P_{NL}}
\end{align*}
Dabei lässt sich die Polarisation in den linearen Anteil $\va{P_L}$ und einen nichlinearen Anteil $\va{P_{NL}}$ zerlegen.
Die partielle Differenzialgleichung lässt sich mit einer zusätzlichen Störung schreiben:
\begin{equation}
  \pdv[2]{E}{x} - \frac{\epsilon_0(1+\chi^{(1)})}{c}\pdv[2]{E}{t} = \frac{1}{c}\pdv[2]{P_{NL}}{t}
  \label{equ:Nl-dgl}
\end{equation}
\paragraph{Erzeugung der zweiten harmonischen}
Betrachtet man nur die nichtlinearität zweiter Ordnung, so verweinfacht sich die Gleichung \ref{equ:Nl-dgl}
zu Gleichung \ref{equ:secondorder_harmonic_generation}
\begin{equation}
  \pdv[2]{E}{x} - \frac{\epsilon_0(1+\chi^{(1)})}{c}\pdv[2]{E}{t} = \frac{\epsilon_0}{c}\pdv[2]{\chi^{(2)}E^2}{t}
  \label{equ:secondorder_harmonic_generation}
\end{equation}
Der Quellterm $\frac{\epsilon_0}{c}\pdv[2]{\chi^{(2)}E^2}{t}$ sorgt für eine induzierte Oszillation mit der doppelten Frequenz $2\omega$.

\subsection{Doppelbrechung}
Doppelbrechung ist der Effekt unterschiedlichder Brechungsindizes $n$ in Abhängigkeit von
der Polarisationsrichtung des Lichtes im Bezug zu einer festen Ebene im doppelbrechenden Kristall. Dieser Effekt ensteht dadurch dass die atomaren Strukturen im Kristall nicht rotationssymmetrisch sind. Mathematisch
lässt sich erläutern, dass die elektrische Suszeptibilität im allgemeinen ein Tensor ist, und somit Richtungseingenschaften aufweist. 
Bei einem optischen Kristall ist es enscheidend, dass die Brechungsindizes und damit die Lichtgeschwindigkeiten
im Material jeweils die selben für die Grund- als auch die Oberschwingung sind. Ist das nicht der Fall,
so interferiert die am Ort $x$ erzeuge zweite harmonische Welle mit der am Ort $x + \triangle x$ erzeugen Welle und löscht sich aus.
Gewisse Materialien wie ein Einkristall von $\beta$-Bariumoxid sind doppelbrechend und optisch aktiv,
und lassen sich damit gut für die Erzeugung der zweiten harmonischen bei gepulsten Lasern nutzen.

\subsection{Autokorrelationsmessung}
Bei Lichtpulsen im Femptisekundenbereich ist jegliche Elektronik zu langsam, um die Halbwertsbreite
eines solchen Lichtpulses zu bestimmen. Um dieses Problem zu umgehen, 'tastet' man den Puls mit sich selber ab.
Dabei wird Lichtpuls mit einem halbdurchlässigen Spiegel gespalten, ein Teil des Pulses durchläuft eine feste Strecke und der andere Teil eine veränderbare Strecke. Danach werden die beiden Pulse wieder zusammengeführt.
Das führt dazu, dass ein genau kontrollierbar zeitlich versetzt zum anderen kommt.
Lässt man beide nun durch einen optisch aktiven Kristall $\beta$-Bariumoxid laufen, so ist die Intensitität der zweiten Harmonischen proportional zur mathematischen Faltung der beiden Peak-Kurven.
\begin{equation}
  W_{2\omega} \propto \int_{-\infty}^{\infty}{I_L(t)I_L(t-t_D)\dd{t}}
  \label{equ:autocorr}
\end{equation}
Dabei beschreibt die Funktion $I_L$ die ursprüngliche Form des elektromagnetischen Pulses.
Verwendet man dabei die Näherung des Peaks als eine Gaussfunktion,
\begin{equation}
  I_L = I_0 \cdot e^{-\big(\frac{2\sqrt{\ln(2)}t}{\tau}\big)^2} = I_0 \cdot e^{-At^2}
  \label{equ:gauss_curve}
\end{equation}
So ergeben Gleichungen \ref{equ:autocorr} und \ref{equ:gauss_curve}:

\begin{align*}
  W_{2\omega} & \propto \int_{-\infty}^{\infty} e^{-At^2} e^{-A(t-t_D)^2} \dd{t} \\
  & = e^{-\frac{A}{2} t_D^2} \cdot \int_{-\infty}^{\infty} e^{-2At^2} \dd{t}
\end{align*}
Damit lässt sich die Halbwertsbreite der Gaussfunktion bestimmen.


%%% Local Variables:
%%% mode: latex
%%% TeX-master: "../Laser"
%%% End:
