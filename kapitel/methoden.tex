\section{Theorie}
\subsection{Funktionsweise eines Lasers}
Man benötigt zwei entscheidende physikalische Aspekte, um einen Laser zu bauen.
Im Folgenden wird ein Zweiniveau-System für ein Elektron mit den Zuständen
$\ket{1}$ und $\ket{2}$ genauer untersucht. \\
Der erste Aspekt ist die induzierte Photonenemission.
Befindet sich ein Elektron im Zustand $\ket{2}$ und wird dieses von einem Photon der Energie $E_{\ket{2}} - E_{\ket{1}}$ getroffen, kann das Elektron in einen niederenergetischen Zustand $\ket{1}$ übergehen. Das emittierte Photon weist dabei die dieselbe Phase, Polarisation, Richtung und vor allem Wellenlänge wie das einfallende Photon auf.
Wenn es nun sehr viele Elektronen im Zustand $\ket{2}$ gibt, so sorgt der Effekt
der induzierten Emission für Photonen mit identischen Eigenschaften. Nun benötigt man
stets mehr Elektronen im Zustand $\ket{2}$ als im Zustand $\ket{1}$, damit es
zur kontinuierlichen induzierten Photoemission kommen kann.
\[
  n_{\ket{2}} > n_{\ket{1}}
\]
Diese Bedingung wird Besetzungsinversion genannt. Das ist der Zweite enscheidende Aspekt.
Mit einem reinen Zweiniveau-System lässt sich die Besetzungsinversion nicht
erreichen, da der Wirkungsquerschnitt zum Anregen der Elektronen derselbe ist,
wie zur induzierten Emission. Man benötigt noch mindestens ein drittes Pumpniveau $\ket{3}$,
auf welches man Elektronen mit Photonen der Energie $\ket{3} - \ket{1}$ anregt, 
von dem diese sich durch schnelle, strahlungsfreie Prozesse auf das langlebigere obere Laserniveau $\ket{2}$ abregen und von
dort durch induzierte Abregung Laserlicht erzeugen.\\
Für einen voll funktionsfähigen Laser werden noch zwei gekrümmte Resonatorspiegel, 
mit Reflektionskoeffizienten $R_1$ und $R_2$ benötigt. Im Laser
muss das Licht stets rückgekoppelt werden, sodass es eine stetige induzierte Anregung gibt.
Für die stetige Rückkopplung müssen die Radien $r_1$ und
$r_2$  die Stabilitätsbedingung erfüllen. Hierbei bezeichnet $L$ den Abstand zwischen den Spiegeln.
\[
   0 <  1 - \frac{L}{r_1} - \frac{L}{r_2} + \frac{L^2}{r_1r_2} < 1 
\]
In dem Resonanzmedium können Photonen absorbiert werden.
Die Verluste pro Strecke werden mit $\delta$ bezeichnet. Die Laseraktivität ist abhängig von den
Verlusten $\delta$ pro Länge, davon wie hoch die Bestzungsinversion $n_{\ket{2}} - n_{\ket{1}}$ ist,
und wie groß der Wirkungsquerschnitt $\sigma$ für die induzierte Emission ist.
Bei einem Durchlauf vom Licht zum ersten Spiegel, zurück zum zweiten Spiegel und wieder zum Anfangspunkt zurück
berechnet man alle Verlust- und Vervielfältigungseffekte vom Licht durch Multiplikation
und erhält für die neue Lichtintensität
\[
I_1 = I_0R_1R_2\cdot \exp{2l(\sigma(n_{\ket{2}} - n_{\ket{1}}) - \delta)}
\]
Für eine Laseraktivität muss gelten $\frac{I_1}{I_0} > 1$, und somit
\[
   \frac{\delta}{\sigma} - \frac{\ln\big(R_1R_0\big)}{2} < n_{\ket{2}} - n_{\ket{1}}
\]

\paragraph{Nd:YFL Laser}
Der im Versuch verwendete Laser hat ein Neodym-Ion als aktives Material.
Dem äußersten Elektron stehen vier Energieniveaus zur Verfügung
Mit Pumplicht aus einer Halbleiterdiode werden die Elektronen im Pumpmaterial
vom niedrigsten Zustand $\ket{1}$ in den Zustand $\ket{4}$ angeregt. Der Übergang
vom Zustand $\ket{4}$ in den langlebigen Zustand $\ket{3}$ erfolgt sehr
schnell. Durch induzierte Lichtemission begeben sich die Elektronen in den kurzlebigen
Zustand $\ket{2}$, von dem sie erneut schnell in den Zustand $\ket{1}$ übergehen.
So ensteht eine große Besetzungsinversion, was wiederum zu einer starken Lichtintensität
nach einem Durchlauf führt.

\subsection{Erzeugung von Ultrakurz-Lichtpulsen}
In dem vorher beschriebenen Resonator aus Spiegeln mit der Länge $L$
entstehen stehende elektromagnetische Longitudinalwellen mit den Wellenlängen
\begin{equation}
  2L = n \cdot \lambda_n
  \label{equ:resonance}
\end{equation}
mit dem Kreisfrequenzabstand $\triangle \omega = \frac{\pi c}{L}$. 
Im Resonator wird aber nur ein Teil aller Wellen, welche die Bedingung \ref{equ:resonance}
erfüllen, verstärkt. 
Erzeugt man akustiche Schwingungen mit der Frequenz $\Omega$ in einem Quarzkristall
und setzt diesen Quarzkristall in die Resonatorstrecke, so werden durch periodische Transmission
der Kreisfrequenz $\omega_n \pm \Omega$. Ist dabei $\Omega = \triangle \omega$
bringt man durch Interferenzeffekte alle Longitudinalmoden in eine feste Phasenbeziehung
und erzeugt damit starke Peaks des elektromagetischen Feldes, Ultrakurzpulse.
Dabei hängt die Dauer und Intensität von der Anzahl der gekoppelten Moden ab.
Je breitbandiger der Laser ist, umso mehr Moden können miteinander gekoppelt werden,
und umso kürzer ist der Peak.

\subsection{Optisch nichtlineare Materialien}
\paragraph{Lineare Optik}
In der klassischen Optik ist die Polarisation eines Dielektrikums linear proportional zur
elektrischen Feldstärke. Dabei ist $\chi_0$ die elektrische Suszeptibilität, also das Verhältnis,
wie sich die Polarisation in Abhängigkeit des elektrischen Feldes im Material ändert.
\begin{equation}
  \va{P}(\va{E}) = \epsilon_0\bigg(\sum_j \chi^{(1)}_{ij} E_j\bigg)
\end{equation}
Die partielle Differenzialgleichung in einer Dimension für das elektrische Feld lautet dabei:
\begin{equation}
  \pdv[2]{E}{x} - \frac{\epsilon_0(1+\chi^{(1)})}{c}\pdv[2]{E}{t} = 0
\end{equation}
Diese Differenzialgleichung wird durch ein elektrisches Feld in Form ebener Wellen gelöst:
\begin{equation}
  E = E_0\exp{i(\omega t - k_{\omega}x)} + C
\end{equation}

\paragraph{Nichtlineare Optik}
In der nichtlinearen Optik wird die durch verschiedende Quanteneffekte beeinflusste Polarisation
eines Mediums mittels eines Taylorpolynoms approximiert.
\begin{align*}
  \va{P}(\va{E}) &= \epsilon_0\bigg(&\sum_j  \chi^{(1)}_{ij} E_j  &+  \sum_{jk} \chi_{ijk}^{(2)} E_j E_k + \sum_{jk\ell} \chi_{ijk\ell}^{(3)} E_j E_k E_\ell + \cdots \bigg)\\
  \va{P}(\va{E}) &= &\va{P_L}\ \ \ \  &+\ \ \ \ \va{P_{NL}}
\end{align*}
Dabei lässt sich die Polarisation in den linearen Anteil $\va{P_L}$ und einen nichlinearen Anteil $\va{P_{NL}}$ zerlegen.
Die partielle Differenzialgleichung lässt sich mit einer zusätzlichen Störung schreiben:
\begin{equation}
  \pdv[2]{E}{x} - \frac{\epsilon_0(1+\chi^{(1)})}{c}\pdv[2]{E}{t} = \frac{1}{c}\pdv[2]{P_{NL}}{t}
  \label{equ:Nl-dgl}
\end{equation}
\paragraph{Erzeugung der zweiten Harmonischen}
Betrachtet man nur die Nichtlinearität zweiter Ordnung, so vereinfacht sich die Gleichung \ref{equ:Nl-dgl}
zu Gleichung \ref{equ:secondorder_harmonic_generation}
\begin{equation}
  \pdv[2]{E}{x} - \frac{\epsilon_0(1+\chi^{(1)})}{c}\pdv[2]{E}{t} = \frac{\epsilon_0}{c}\pdv[2]{\chi^{(2)}E^2}{t}
  \label{equ:secondorder_harmonic_generation}
\end{equation}
Der Quellterm $\frac{\epsilon_0}{c}\pdv[2]{\chi^{(2)}E^2}{t}$ sorgt für eine induzierte Oszillation mit der doppelten Frequenz $2\omega$.

\subsection{Doppelbrechung}
Doppelbrechung ist der Effekt unterschiedlicher Brechungsindizes $n$ in Abhängigkeit von
der Polarisationsrichtung des Lichtes im Bezug zu einer festen Ebene im doppelbrechenden Kristall. Dieser Effekt ensteht dadurch, dass die atomaren Strukturen im Kristall nicht rotationssymmetrisch sind. Mathematisch
lässt sich erläutern, dass die elektrische Suszeptibilität im Allgemeinen ein Tensor ist, und somit Richtungseingenschaften aufweist. 
Bei einem optischen Kristall, der für nichtlineare optische Effekte genutzt werden soll, ist es entscheidend, dass die Brechungsindizes und damit die Lichtgeschwindigkeiten
im Material sowohl für das einfallende als auch für das in dem nichtlinearen Prozess erzeugte Licht, in diesem Fall die zweite Harmonische, gleich sind, was in Abschnitt 5.8 näher erläutert wird.
Gewisse Materialien wie ein $\beta$-Bariumborateinkristall sind doppelbrechend und optisch aktiv,
und lassen sich damit gut für die Erzeugung der zweiten Harmonischen bei gepulsten Lasern nutzen.

\subsection{Autokorrelationsmessung}
Bei Lichtpulsen im Picosekundenbereich ist jegliche Elektronik zu langsam, um die zeitliche Halbwertsbreite
eines solchen Lichtpulses zu bestimmen. Um dieses Problem zu umgehen, 'tastet' man den Puls mit sich selber ab.
Dabei wird der Lichtpuls mit HIlfe eines halbdurchlässigen Spiegels gespalten, ein Teil des Pulses durchläuft eine feste Strecke und der andere Teil eine veränderbare Strecke. Danach werden die beiden Pulse wieder zusammengeführt.
Das führt dazu, dass ein Puls genau kontrollierbar zeitlich versetzt zum anderen kommt.
Lässt man beide nun durch einen optisch aktiven $\beta$-Bariumoxidkristall laufen, so ist die Intensitität der zweiten Harmonischen proportional zur mathematischen Faltung der beiden Peak-Kurven.
\begin{equation}
  W_{2\omega} \propto \int_{-\infty}^{\infty}{I_L(t)I_L(t-t_D)\dd{t}}
  \label{equ:autocorr}
\end{equation}
Dabei beschreibt die Funktion $I_L$ die ursprüngliche Form des elektromagnetischen Pulses.
Verwendet man dabei die Näherung des Peaks als eine Gaußfunktion,
\begin{equation}
  I_L = I_0 \cdot e^{-\big(\frac{2\sqrt{\ln(2)}t}{\tau}\big)^2} = I_0 \cdot e^{-At^2}
  \label{equ:gauss_curve}
\end{equation}
so ergeben Gleichungen \ref{equ:autocorr} und \ref{equ:gauss_curve}:

\begin{align}
\begin{split}
  W_{2\omega} & \propto \int_{-\infty}^{\infty} e^{-At^2} e^{-A(t-t_D)^2} \dd{t} \\
& = e^{-\frac{A}{2} t_D^2} \cdot \int_{-\infty}^{\infty} e^{-2At^2} \dd{t}
\label{eq:Autokorr}
\end{split}
\end{align}
Damit lässt sich die Halbwertsbreite der Gaufunktion bestimmen.


%%% Local Variables:
%%% mode: latex
%%% TeX-master: "../Laser"
%%% End:
