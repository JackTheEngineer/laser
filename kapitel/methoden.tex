\section{Theorie}
\subsection{Funktionsweise eines Lasers}
Man benötigt zwei entscheidende Physikalische Aspekte, um einen Laser zu bauen.
Im folgenden wird ein Zweiniveausystem für ein Elektron mit den Zuständen
$\ket{1}$ und $\ket{2}$ genauer Untersucht. \\
Der erste Aspekt ist die induzierte Photonenemission. Wenn es ein Elektron in einem
angeregten Zustand $\ket{2}$ gibt, und es mit einem Photon interagiert, welches die
Differenzenergie der Zustände $\ket{2} - \ket{1}$ trägt, dann regt sich das
Elektron ab und emittiert dabei ein Photon welches deselbe Phase,
Polarisation und Richtung hat, wie das auf das Elektron eingefallene Photon.
Wenn es nun sehr viele Elektronen im Zustand $\ket{2}$ gibt, so sorgt der Effekt
der induzierten Emission für Photonen mit identischen Eigenschaften. Nun benötigt man
stets mehr Elektronen im Zustand $\ket{2}$ als im Zustand $\ket{1}$, damit es
zur kontinuierlichen induzierten Photoemission kommen kann.
\[
  n_{\ket{2}} > n_{\ket{1}}
\]
Diese Bedingung heißt Besetzungsinversion. Das ist der Zweite enscheidene Aspekt.
Mit einem reinen Zweiniveausystem lässt sich die Besetzungsinversion nicht
erreichen, da der Wirkungsquerschnitt zum anregen der Elektronen derselbe ist,
wie zum induzierten abregen. Man benötigt noch mindestens ein drittes Pumpniveau $\ket{3}$,
auf welches man Elektronen mit Photonen der Energie $\ket{3} - \ket{1}$ befördert, 
damit sich möglichst schnell auf das langlebigere Niveau $\ket{2}$ abregen und von
dort durch induzierte Abregung Laserlicht erzeugen.\\
Für einen voll funktionsfähigen Laser benötigt es noch zwei gekrümmte Resonatorspiegel, 
mit reflektionskoeffizienten $R_1$ und $R_2$. Im Laser
muss das Licht stets rückgekoppelt werden, sodass es eine stetige induzierte Anregung gibt.
Für die stetige Rückkopplung müssen die Radien $r_1$ und
$r_2$  die Stabilitätsbedingung erfüllen. Hierbei bezeichnet $L$ den Abstand zwischen den Spiegeln.
\[
   0 <  1 - \frac{L}{r_1} - \frac{L}{r_2} + \frac{L^2}{r_1r_2} < 1 
\]
In dem Resonanzmedium können Photonen absorbiert werden.
Die Verluste pro Strecke werden mit $\delta$ bezeichnet. Die Laseraktivität ist abhängig von den
Verlusten $\delta$ pro, davon wie groß die Bestzungsinversion $n_{\ket{2}} - n_{\ket{1}}$ ist,
und wie groß der Wirkungsquerschnitt $\sigma$ für die induzierte Emission ist.
Bei einem Durchlauf vom Licht zum ersten Spiegel, zurück zum zweiten Spiegel und wieder zum Anfangspunkt zurück
berechnet man alle Verlust- und Vervielfältigungseffekte vom Licht durch multiplikation
und erhält für die neue Lichtintensität
\[
I_1 = I_0R_1R_2\cdot \exp{2l(\sigma(n_{\ket{2}} - n_{\ket{1}}) - \delta)}
\]
Für eine Laseraktivität muss gelten $\sfrac{I_1}{I_0} > 1$, und somit
\[
   \frac{\delta}{\sigma} - \frac{ln\big(R_1R_0\big)}{2} < n_{\ket{2}} - n_{\ket{1}}
\]

\paragraph{Nd:YFL Laser}
Der im Versuch Verwendete Laser hat ein Neodym-Ion als aktives Material.
Das äußerste elektron hat dabei vier Energieniveaus zur Verfügung stehen.
Mit Pumplicht aus einer Halbleiterdiode werden die Elektronen im Pumpmaterial
vom niedrigsten Zustand $\ket{1}$ in den Zustand $\ket{4}$ angeregt. Der Übergang des
vom Zustand $\ket{4}$ in den langlebigen Zustand $\ket{3}$ erfolgt sehr
schnell. Durch induzierte Lichtemission begeben sich die Elektronen in den kurzlebigen
Zustand $\ket{2}$, von dem sie erneut schnell in den Zustand $\ket{1}$ übergehen.
So ensteht eine grosse Besetzungsinversion, wiederum zu einer starken Lichintensität
nach einem Durchlauf führt.

\subsection{Erzeugung von Ultrakurz-Lichpulsen}
In dem vorher beschriebenen Resonator aus Spiegeln mit der Länge $L$
entstehen elektromagnetische stehende Longitudinalwellen mit den Wellenlängen
\begin{equation}
  2L = n\lambda_n
  \label{equ:resonance}
\end{equation}
mit dem Kreisfrequenzabstand $\triangle \omega = \sfrac{\pi c}{L}$.
Im Resonator wird aber nur ein Teil aller Wellen welche die Bedingung \ref{equ:resonance}
erfüllen, verstärkt. Durch periodische Transmission des Elektromagnetischen Feldes mit einer
Frequenz $\Omega$ durch einen Quarzkristall, erzeugt man zusätzliche Seitenbänder mit
der Kreisfrequenz $\omega \pm \Omega$.
Durch Interferenzeffekte bringt man somit alle Longitudinalmoden in eine feste Phasenbeziehung
und erzeugt damit starke Peaks des elektromagetischen Feldes, Ultrakurzpulse.
Dabei hängt die Dauer und intensität von der Anzahl der gekoppelten Moden ab.
Je breitbandiger der Laser ist, umso mehr Moden können miteinander gekoppelt werden,
und umso kürzer ist der Peak.

\subsection{Optisch nichtlineare Materialien}



%%% Local Variables:
%%% mode: latex
%%% TeX-master: "../Laser"
%%% End:
